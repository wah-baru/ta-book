% Atur variabel berikut sesuai namanya

% nama
\newcommand{\name}{Ja'far Shadiq}
\newcommand{\authorname}{Shadiq, Ja'far}
\newcommand{\nickname}{Ja'far}
\newcommand{\advisor}{Dion Hayu Fandiantoro, S.T., M.Eng.}
\newcommand{\coadvisor}{Eko Pramunanto, S.T., M.T.}
\newcommand{\examinerone}{Dr. Eko Mulyanto Yuniarno, S.T., M.T.}
\newcommand{\examinertwo}{Dr. Arief Kurniawan, S.T., M.T.}
\newcommand{\examinerthree}{Dr. Diah Puspito Wulandari, S.T., M.Sc.}
\newcommand{\examinerfour}{Arta Kusuma Hernanda, S.T., M.T.}
\newcommand{\headofdepartment}{Dr. Supeno Mardi Susiki Nugroho, S.T., M.T.}

% identitas
\newcommand{\nrp}{0721 19 4000 0023}
\newcommand{\advisornip}{1994202011064}
\newcommand{\coadvisornip}{19661203 199412 1 001}
\newcommand{\examineronenip}{19680601 199512 1 009}
\newcommand{\examinertwonip}{19740907 200212 1 001}
\newcommand{\examinerthreenip}{19801219 200501 2 001}
\newcommand{\examinerfournip}{1996202311024}
\newcommand{\headofdepartmentnip}{19700313 199512 1 001}

% judul
\newcommand{\tatitle}{Sistem Pemantauan Utilisasi Ruang Perkuliahan Berbasis Okupansi Menggunakan Kamera}
\newcommand{\engtatitle}{\emph{ANTI-GRAVITY BASED ENERGY CALCULATION ON OUTER SPACE ROCKETS}}

% tempat
\newcommand{\place}{Surabaya}

% jurusan
\newcommand{\studyprogram}{Teknik Komputer}
\newcommand{\engstudyprogram}{Computer Engineering}

% fakultas
\newcommand{\faculty}{Teknologi Electro}
\newcommand{\engfaculty}{Intelligent Electrical and Informatics Technology}

% singkatan fakultas
\newcommand{\facultyshort}{FTEIC}
\newcommand{\engfacultyshort}{F-ELECTICS}

% departemen
\newcommand{\department}{Teknik Komputer}
\newcommand{\engdepartment}{Computer Engineering}

% kode mata kuliah
\newcommand{\coursecode}{TD123456}
